\subsection{A bit of history}

It is a classical and well-known result that compact two-sided cancellative semigroups are topological groups (e.g. \cite{gelbaum1951embedding}). 
In 1955, Wallace \cite{wallace1955structure} has asked if every countably compact two-sided cancellative semigroup is a topological group.
This question has been known as the Wallace's problem, and an example yielding a negative answer is known as a Wallace semigroup.
The first consistent example of such a semigroup, due to D. Robbie and S. Svetlichny, appeared forty-one years later in the Proceedings of the American Mathematical Society \cite{robbie1996answer}, under the Continuum Hypothesis (CH).
Since then, several examples of Wallace semigroups have been constructed from Martin's axiom and selective ultrafilters.

Grant \cite{grant1993sequentially} showed that sequentially compact two-sided cancellative semigroups are topological groups and asked whether $p$-compact two-sided cancellative semigroups are topological groups.
Recall that a topological space is $p$-compact for some free ultrafilter $p$ if and only if its $2^\mathfrak c$-th power is countably compact.
Taking that into account, in \cite{grant1993sequentially}, Grant also asked whether every two-sided cancellative topological semigroup with countably compact square is a topological group.
The first question was answered positively in \cite{T96}, in ZFC.
A consistent counter-example for the second question was given in \cite{Boero}.
Thus, it natural to ask what is the highest degree of countably compactness which still admits a Wallace semigroup.

In \cite{Boero}, Boero and Tomita consistently produced a group topology for $\mathbb Z^{(\mathfrak c)}$ whose square is countably compact using $\mathfrak c $ selective ultrafilters and, as a corollary, they showed that its subsemigroup $\mathbb N^{(\mathfrak c)}$ with the subspace topology was a Wallace semigroup whose square was still countably compact.

Moreover, in \cite{tomita2015group}, Tomita consistently proved that $\mathbb Z^{(\mathfrak c)}$ admits a group topology whose all finite powers are countably compact. For that, the existence of $\mathfrak{c}$ pairwise incomparable selective ultrafilters was assumed - an hypothesis weaker than both the Continuum Hypothesis and Martin's Axiom. The author tried to consider the subsemigroup $\mathbb N^{(\mathfrak c)}$, but he was unable to guarantee that its finite powers are countably compact as well. With that in mind, he asked whether there exists a Wallace semigroup whose all finite powers are countably compact (Problem 9.4.).

In this paper, we consistently answer the preceding question positively by showing that,assuming the existence of $\mathfrak c$ incomparable selective ultrafilters, the semigroup $\mathbb N\oplus \mathbb Z^{(\mathfrak c)}$ can be endowed with a Hausdorff topology whose all finite powers are countably compact. To do so, we employ the technique of \textit{integer stacks}, which is revised in Section 2. This method was introduced by the third author in \cite{tomita2015group} to prove a similar result to $\mathbb Z^{(\mathfrak c)}$. We note that this is the first time this technique is used with semigroups, even thought it was not developed with this intent. Thus, we consider this paper to be evidence that integer stacks are more flexible than previously believed.

We also show (in ZFC) that this semigroup cannot have a topology whose $\omega$-th power is countably compact. Note that this semigroup is also a subsemigroup of $\mathbb Z^{(\mathfrak c)}$.

During the development of this paper, we also could not make an example out of $\mathbb N^{(\mathfrak c)}$. Thus, the existence of a semigroup topology in $\mathbb N^{(\mathfrak c)}$ whose every finite power is countably compact remains an open problem.


