
\section{A briefing on stacks}

In \cite{tomita2015group}, a structure called \textit{stack} has been defined to produce a group topology for $\mathbb Z^{(\mathfrak c)}$ without convergent subsequences such that every finite power is countably compact, with the aid of selective ultrafilters.

A stack is, as we shall review, a finite collection of sequences of elements of $\mathbb Z^{(\mathfrak c)}$ satisfying certain properties and having some extra structure.

Roughly speaking, given a finite collection of sequences in $\mathbb Z^{(\mathfrak c)}$, we \emph{associate} it to a stack in a way so that if the sequences in the stack have $p$-limits in some group topology, so do the original collection.
Thus, if for every stack, we can guarantee the existence of a free ultrafilter $p$ such that the sequences in the stack have $p$-limits, we guarantee that all the finite powers of $\mathbb Z^{(\mathfrak c)}$ are countably compact.
In order for that, a family of homomorphisms that separates points (to ensure Hausdorffness) is constructed.
Particular properties of a fixed family of ultrafilters may be used to construct such homomorphisms, and that is where the assumption of the existence of $\mathfrak c$ incomparable selective ultrafilters comes into play in \cite{tomita2015group}.
The construction of such homomorphisms is very convoluted, and the interested reader may refer to \cite{tomita2015group} for details.
Here, we will review the definition of stacks and state all the known results about them that will be used in this paper.
    
    \begin{defin} 
    Let $A$ be an infinite subset of $\omega$.
    An integer stack $\mathcal S$ on $A$ is composed by
		\begin{itemize}
			\item natural numbers $s,t,M$, positive integers $r_i$ for each $0 \leq i < s$, and positive integers $r_{i,j}$ for each $0 \leq i < s$ and $0 \leq j < r_i$,
			\item functions $f_{i,j,k} \in ( \mathbb Z^{(\mathfrak c)})^{A}$ for each $0 \leq i < s$, $0 \leq j < r_i$, $0 \leq k < r_{i,j}$ and $g_l \in ( \mathbb Z^{(\mathfrak c)})^{A}$ for each $0 \leq l < t$, 
			\item sequences $\xi_i \in \mathfrak c^{A}$ for each $0 \leq i < s$ and $\mu_l \in \mathfrak c^{A}$ for $0 \leq l < t$, 
			\item real numbers $\theta_{i,j,k}$ for each $0 \leq i < s$, $0 \leq j < r_i$, $0 \leq k < r_{i,j}$,
		\end{itemize}
	satisfying the following
		\begin{enumerate}[label=\roman*)]
			\item $\mu_l(n) \in \supp g_l(n)$ for every $n \in A$,
            \item $\mu_{l^*}(n) \notin \supp g_l(n)$ for each $n \in A$ and $0 \leq l^* < l < t$,
			\item  the elements of $\lbrace \mu_l(n) \mid 0 \leq l < t, n \in A \rbrace$ are pairwise distinct,
			\item $\vert g_l(n) \vert \leq M$ for each $n \in A$ and $0 \leq l < t$,
			\item $(\theta_{i,j,k} \mid 0 \leq k < r_{i,j} )$ is injective and linearly independent (as a $\mathbb Q$-vector space) for each $0 \leq i < s$ and $0 \leq j < r_i$,
            \item $\left( \frac{f_{i,j,k}(n)(\xi_i(n))}{f_{i,j,0}(n)(\xi_i(n))} \right)_{n \in A} \to \theta_{i,j,k}$ for each $0 \leq i < s$, $0 \leq j < r_i$ and $0 \leq k < r_{i,j}$,
			\item $\left( \vert f_{i,j,k}(n)(\xi_i(n)) \vert \right)_{n \in A}$ converges monotonically to $+\infty$ for each $0 \leq i < s$, $0 \leq j < r_i$ and $0 \leq k < r_{i,j}$,
			\item $\vert f_{i,j,k}(n)(\xi_i(n)) \vert > \vert f_{i,j,k^*}(n)(\xi_i(n)) \vert$ for each $n \in A$, $i < s$, $j < r_i$ and $0 \leq k < k^* < r_{i,j}$,
			\item $\left( \frac{f_{i,j,k}(n)(\xi_i(n))}{f_{i,j^*,k^*}(n)(\xi_i(n))} \right)_{n \in A}$  converges monotonically to $0$ for each $0 \leq i < s$, $0 \leq j^* < j < r_i$, $0 \leq k < r_{i,j}$ and $0 \leq k^* < r_{i,j^*}$,
			\item $\lbrace f_{i,j,k}(n)(\xi_{i^*}(n)) \mid n \in A \rbrace \subset [-M, M]$ for each $0 \leq i^* < i < s$, $0 \leq j < r_i$ and $0 \leq k < r_{i,j}$.
		\end{enumerate}
    The support of an integer stack $\mathcal S$ on $A$ is defined by the union of the support of each $f_{i,j,k}(n)$ and $g_l(n)$ (for $n \in A$). 
	\end{defin}
     
    \begin{defin}
    Let $\mathcal S$ be an integer stack on $A$ and $N$ be a positive integer. The $N$th root of $\mathcal S$, denoted by $\frac{1}{N}. \mathcal S$, is obtained by replacing $f_{i,j,k}$ by $\frac{1}{N}. f_{i,j,k}$ for each $i < s$, $j < r_i$ and $k < r_{i,j}$, replacing $g_l$ by $\frac{1}{N}. g_{l}$ for each $l < t$, and keeping the rest of the structure of $\mathcal{S}$. 

    A \textit{stack} is a structure of the form $\frac1N \mathcal S$, where $\mathcal S$ is an integer stack and $N$ is a positive integer.
    \end{defin}

   
    The following proposition, originally from \cite[{Lemma 7.1.}]{tomita2015group}. It has been restated in \cite{bellini2023countably} with more specific statements. The latter version, in the notation of the current paper, reads as follows:

    \begin{lem}[Lemma 5.4., \cite{bellini2023countably}]\label{stack}\label{tomita2015grouplemma7.1}
	Let $p$ be a selective ultrafilter, $m \in \omega$ and $(h_i: i <m)$ be sequences in $\mathbb Z^{(\mathfrak c)}$ which are $p$-independent mod constants (with respect to $G=\mathbb Z^{(\mathfrak c)}$).
 
 Then there exists $A\in p$, a positive integer $N$ and a integer stack $\calS$ on~$A$ 
	such that, for each $i<m$, $h_i|_A$  is an 
		integer combination of the elements of the sequences of the stack $\frac{1}{N} \calS$ restricted to $A$. 
		On the other hand, each sequence of the integer stack~$\calS$ is 
		an integer combination of $(h_i|A)_{i<m}$

	
	We will say in this case that the finite sequence 
	$( h_0, \ldots , h_{m-1})$ is $p$-associated to $(\calS, N, A)$.
	
\end{lem}

	It is worth mentioning that the hypothesis ``$(h_i: i<m)$ are sequences in $\mathbb Z^{(\mathfrak c)}$ which are $p$-independent mod constants'', in \ref{tomita2015grouplemma7.1}, reads as ``$(h_i:i<m)$  are sequences in $\mathbb Z^{(\mathfrak c)}$ such that the family $([h_i]_p:i<m)$ united with $([\vec \chi_{\alpha}]_p: \alpha<\mathfrak c)$ is linearly independent over $\mathbb Q$''. However, both these hypothesis are easily seen to be equivalent.

    We have the following:

    \begin{remark}\label{suptRemark}
        In the notation of Lemma \ref{tomita2015grouplemma7.1}, if $(h_i)_{i<m}$ is $p$-associated to $(\mathcal S, N, A)$, then  $\bigcup_{n \in A, i<m}\supp h_i(n)$ is the union of the supports of the elements of the integer stack $\mathcal S$. Also, if $M|N$, $(\mathcal S, M, A)$ is also associated to $(h_i)_{i<m}$.
    \end{remark}

    The importance of stacks with respect to countably compactness comes from the following result:

    

    \begin{lem}[Lemma 8.4, \cite{tomita2015group}] \label{tomita2015grouplema8.4}
    Let  $(p_m : m \in \omega)$ be a family of incomparable selective ultrafilters and $( N_m : m \in \omega)$ be an enumeration of the positive integers such that each such integer is appears infinitely often.
    
    Let $\mathcal S_m$ be an integer stack on $A_m$ with $A_m \in p_m$ for each $m \in \omega$. In addition, consider $s^m$, $t^m$, $r^m_i$, $r^m_{i,j}$, $M^m$, $r^m$, $L^m$, $f^m_{i,j,k}$ and $g^m_l$ as part of the structure of $\mathcal S_m$.

    For any family $( c)^\frown(c^m_{i,j,k} : m < \omega, i < s^m, j < r^m_i, k < r^m_{i,j})^\frown (d^m_l \mid m < \omega, l < t^m)$ of $\mathbb Z^{(\mathfrak c)}$ with $c \neq 0$,  there exists a homomorphism $\phi : \mathbb{Q}^{(\mathfrak c)} \rightarrow \mathbb{\mathbb T}$ such that
        \begin{enumerate}
            \item[a)] $\phi(c) \neq 0$,
            \item[b)] $p_{m}$-$ \lim\limits_{n \to \infty} \phi \left( \frac{1}{N_m} f^m_{i,j,k}(n) \right) = \phi \left(  c^m_{i,j,k} \right)$ for any $i < s^m$, $j < r^m_i$, $k < r^m_{i,j}$ and $m \in \omega$,
            \item[c)] $p_{m}$-$ \lim\limits_{n \to \infty} \phi \left( \frac{1}{N_m} g^m_{l}(n) \right) = \phi ( d^m_{l} )$ for any $l < t^m$ and $m \in \omega$.
        \end{enumerate}
    \end{lem}

    
   We will need the following definition related to stacks:


    \begin{defin} 
    For $\gamma=0, 1$, let $\mathcal S_\gamma$, be the stacks defined by $s^\gamma$, $t^\gamma$, $M^\gamma$, $(r_i^\gamma: i<s^\gamma)$, $(r_{i, j}^\gamma: 0\leq i<s^\gamma, 0\leq j<r_i^\gamma)$, $(f_{i, j, k}^{\gamma}: i<s^\gamma, j<r_i^\gamma, k<r_{i, j}^\gamma)$, $(\xi_i^\gamma: i<s^\gamma$, $(\mu_l^\gamma: l<t^\gamma)$ and $(\theta_{i, j, k}^\gamma: i<s^\gamma, j<r_i^\gamma, k<r_{i, j}^\gamma)$.
    
    Assume $\mathcal S_0$ and $\mathcal S_1$ have disjoint supports. The union of $\mathcal S_0$, $\mathcal S_1$ is the stack $\mathcal S_0\sqcup \mathcal S_1$ defined as follows:

    Let $t=t^0+t^1$, $s=s^0+s^1$ and $M=M^0+M^1$. Let $g_l=g^0_l$ for $l<t_o$ and $g_l=g^1_{l-t_0}$ if $t^0\leq l <t^0+t^1$. If $i<s^0$ define $r_i=r^0_i$, $r_{i,j}=r^0_{i,j}$ for $j<r_i$ and $f_{i,j,k}=f^0_{i,j,k}$.

    If $s^0\leq i< s^0+s^1$ define $r_i=r^1_{i-s^0}$, $r_{i,j}=r^1_{i-s^0,j}$ for $j<r_i$ and $r_{i,j}=r^1_{i-s^0,j}$ for $j< r_{i}=r^1_{i-s^0}$ and $f_{i,j,k}=f^1_{i-s^0,j,k}$ for $j<r_i$ and $k<r_{i,j}$. Define $\xi_l$'s and $\theta_{i,j,k}$ simillarly.
    \end{defin}

    It is straightforward to verify that $\mathcal S_0\sqcup \mathcal S_1$ is an integer stack. We note that this easy verification is the only verification in this paper for which looking at bullet points i)-x) is needed, and we leave the proof as an easy exercise. Observe, however, that the hypothesis of disjoint supports is relevant. 
